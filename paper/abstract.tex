Este artigo avalia o algoritmo xNES para otimização funções reais presentes no conjunto de \textit{benchmark} do CEC 2014.
Nele é analisado o impacto da variação de seus parâmetros e o comparam com técnicas de estado da arte em otimização real, CMA-ES.
São destacas as diferentes características entre os algoritmos e funções custo, de modo a examinar efetividade de cada método em diversos cenários.
