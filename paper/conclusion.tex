Concluímos que a seleção de parâmetros do xNES foi capaz de apresentar resultados equivalentes ou superiores aos
parâmetros padrão, sendo o tamanho da população consideravelmente maior do que o proposto originalmente.
Comparando o xNES com parâmetros selecionados com o CMA-ES observamos que em praticamente todos os casos os algoritmos
obtiveram resultados equivalentes, ficando dentro da margem de erro.

Os algoritmos apresentaram performance muito semelhantes entre si, este fato pode ser explicado devido a semelhança entre
os dois algoritmos, sendo o método de estimação da matriz de covariância sua única diferença relevante.

Os algoritmos se mostraram eficientes computacionalmente, necessitando de pouco tempo de processamento e memória.
Desta forma, avalia-se que são boas escolhas para otimização de funções reais.
