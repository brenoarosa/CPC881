Algoritmos evolucionários são comumente aplicados na otimizações de funções caixa preta.
Destaca-se a família de algorítmos Estratégia Evolucionária, desenvolvida inicialmente por Rechenberg~\cite{rechenberg65}
nos meados dos anos 60.
Esses algoritmos percorrem o espaço de busca através de passos aleatórios de tamanho adaptativo, utilizando os métodos de
evolução comuns a outros algoritmos evolucionários, como: seleção, mutação e recombinação.

Dentre esses métodos, o CMA-ES~\cite{hansen01} é um dos algoritmos mais competitivos atualmente, no qual o passo de
mutação é definido a partir de uma matriz de covariância adaptativa, tornando o método invariante a certas transformações
lineares do espaço.
Outros algoritmos semelhantes são os \textit{Natural Evolution Strategies} (NES)~\cite{wierstra08} e posteriormente
os \textit{Exponential Natural Evolution Strategies} (xNES)~\cite{glasmachers10}, que se diferenciam por estimar a matriz de
covariância baseada no gradiente natural do erro, mantendo assim a invariância a transformações aprensentada pelo CMA-ES.

Este artigo avalia o algoritmo xNES, variando seus parâmetros e comparando sua performance a partir de um conjunto de
funções de avaliação pré-selecionado.
Essas funções de \textit{benchmark} foram retiradas da Competição de Otimização Real Mono Objetivo, realizada no Congresso
de Computação Evolucionária do ano de 2014~\cite{liang13}.

A estrutura deste artigo será disposta da seguinte forma: a seção~\ref{sec:methodology} descreverá o método utilizado.
Posteriormente, na seção~\ref{sec:results}, serão apresentos os resultados e suas análises. Por fim, a
seção~\ref{sec:conclusion} contará com a conclusão do trabalho.
